% Label Noise Robustness Table
% Shows SPI prediction accuracy under various label noise levels

\begin{table}[t]
\centering
\caption{Label Noise Robustness: SPI Predictions Under Noisy Labels}
\label{tab:label_noise_robustness}
\small
\begin{tabular}{l|ccccc|c}
\toprule
\textbf{Dataset} & \textbf{0\%} & \textbf{5\%} & \textbf{10\%} & \textbf{20\%} & \textbf{30\%} & \textbf{Correct} \\
\midrule
Cora (h=0.81) & \checkmark & \checkmark & \checkmark & \checkmark & \checkmark & 5/5 \\
CiteSeer (h=0.74) & \checkmark & \checkmark & \checkmark & \checkmark & \checkmark & 5/5 \\
PubMed (h=0.80) & \checkmark & \checkmark & \checkmark & \checkmark & \checkmark & 5/5 \\
\midrule
\multicolumn{6}{r|}{\textbf{Overall SPI Accuracy:}} & \textbf{100\%} (15/15) \\
\bottomrule
\end{tabular}
\vspace{1mm}

\footnotesize
\textit{Note:} \checkmark indicates SPI correctly predicted GNN vs MLP winner.
Key finding: GCN shows increasing advantage as noise increases (denoising effect through aggregation).
At 30\% noise, Cora GCN advantage grows from +12.5\% to +21.4\%.
\end{table}
